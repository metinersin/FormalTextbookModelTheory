
\section{Vectors}
%%% Really trivial stuff about numbers, vectors, etc.

\begin{lemma}
    \label{lem:two-vector-notation}
    \lean{Matrix.Vector_eq_VecNotation₂}
    \leanok
    For any vector $v : \Fin 2 \to \bbn$, we have $v = ![v(0), v(1)]$.
\end{lemma}
\begin{proof}
    \leanok
\end{proof}
%%%

\section{Theories}
%%% Really trivial stuff about theories

\begin{lemma}
    \label{lem:linear-order-theory-in-dlo}
    \lean{FirstOrder.Language.Order.linearOrderTheory_subset_dlo}
    \leanok
    The theory of linear orders is a subset of DLO.
\end{lemma}
\begin{proof}
    \leanok
\end{proof}

%%%% LinearOrder
\begin{lemma}
    \label{lem:reflexive-in-linear-order-theory}
    \lean{FirstOrder.Language.Order.reflexive_in_linearOrderTheory}
    \leanok
    The theory of linear orders contains the reflexivity sentence.
\end{lemma}
\begin{proof}
    \leanok
\end{proof}

\begin{lemma}
    \label{lem:transitive-in-linear-order-theory}
    \lean{FirstOrder.Language.Order.transitive_in_linearOrderTheory}
    \leanok
    The theory of linear orders contains the transitivity sentence.
\end{lemma}
\begin{proof}
    \leanok
\end{proof}

\begin{lemma}
    \label{lem:antisymmetric-in-linear-order-theory}
    \lean{FirstOrder.Language.Order.antisymmetric_in_linearOrderTheory}
    \leanok
    The theory of linear orders contains the antisymmetry sentence.
\end{lemma}
\begin{proof}
    \leanok
\end{proof}

\begin{lemma}
    \label{lem:total-in-linear-order-theory}
    \lean{FirstOrder.Language.Order.total_in_linearOrderTheory}
    \leanok
    The theory of linear orders contains the totality sentence.
\end{lemma}
\begin{proof}
    \leanok
\end{proof}

\begin{lemma}
    \label{lem:models-of-linearOrderTheory-realize-reflexive}
    \lean{FirstOrder.Language.Order.realize_reflexive_of_model_linearOrderTheory}
    \leanok
    Models of the theory of linear orders realize the reflexivity sentence.
\end{lemma}
\begin{proof}
    \leanok
    \uses{lem:reflexive-in-linearOrderTheory}
\end{proof}

\begin{lemma}
    \label{lem:models-of-linearOrderTheory-realize-transitive}
    \lean{FirstOrder.Language.Order.realize_transitive_of_model_linearOrderTheory}
    \leanok
    Models of the theory of linear orders realize the transitivity sentence.
\end{lemma}
\begin{proof}
    \leanok
    \uses{lem:transitive-in-linearOrderTheory}
\end{proof}

\begin{lemma}
    \label{lem:models-of-linearOrderTheory-realize-antisymmetric}
    \lean{FirstOrder.Language.Order.realize_antisymmetric_of_model_linearOrderTheory}
    \leanok
    Models of the theory of linear orders realize the antisymmetry sentence.
\end{lemma}
\begin{proof}
    \leanok
    \uses{lem:antisymmetric-in-linearOrderTheory}
\end{proof}

\begin{lemma}
    \label{lem:models-of-linearOrderTheory-realize-total}
    \lean{FirstOrder.Language.Order.realize_total_of_model_linearOrderTheory}
    \leanok
    Models of the theory of linear orders realize the totality sentence.
\end{lemma}
\begin{proof}
    \leanok
    \uses{lem:total-in-linearOrderTheory}
\end{proof}

%%%%

%%%% DLO
\begin{lemma}
    \label{lem:reflexive-in-dlo}
    \lean{FirstOrder.Language.Order.reflexive_in_dlo}
    \leanok
    The theory DLO contains the reflexivity sentence.
\end{lemma}
\begin{proof}
    \leanok
    \uses{lem:linear-order-theory-in-dlo, lem:reflexive-in-linear-order-theory}
\end{proof}

\begin{lemma}
    \label{lem:transitive-in-dlo}
    \lean{FirstOrder.Language.Order.transitive_in_dlo}
    \leanok
    The theory DLO contains the transitivity sentence
\end{lemma}
\begin{proof}
    \leanok
    \uses{lem:linear-order-theory-in-dlo, lem:transitive-in-linear-order-theory}
\end{proof}

\begin{lemma}
    \label{lem:antisymmetric-in-dlo}
    \lean{FirstOrder.Language.Order.antisymmetric_in_dlo}
    \leanok
    The theory DLO contains the antisymmetry sentence.
\end{lemma}
\begin{proof}
    \leanok
    \uses{lem:linear-order-theory-in-dlo, lem:antisymmetric-in-linear-order-theory}
\end{proof}

\begin{lemma}
    \label{lem:total-in-dlo}
    \lean{FirstOrder.Language.Order.total_in_dlo}
    \leanok
    The theory DLO contains the totality sentence.
\end{lemma}
\begin{proof}
    \leanok
    \uses{lem:linear-order-theory-in-dlo, lem:total-in-linear-order-theory}
\end{proof}

\begin{lemma}
    \label{lem:models-of-dlo-realize-reflexivity}
    \lean{FirstOrder.Language.Order.realize_reflexive_of_model_dlo}
    \leanok
    Models of the theory DLO realize the reflexivity sentence.
\end{lemma}
\begin{proof}
    \leanok
    \uses{lem:reflexive-in-dlo}
\end{proof}

\begin{lemma}
    \label{lem:models-of-dlo-realize-transitivity}
    \lean{FirstOrder.Language.Order.realize_transitive_of_model_dlo}
    \leanok
    Models of the theory DLO realize the transitivity sentence.
\end{lemma}
\begin{proof}
    \leanok
    \uses{lem:transitive-in-dlo}
\end{proof}

\begin{lemma}
    \label{lem:models-of-dlo-realize-antisymmetric}
    \lean{FirstOrder.Language.Order.realize_antisymmetric_of_model_dlo}
    \leanok
    Models of the theory DLO realize the antisymmetry sentence.
\end{lemma}
\begin{proof}
    \leanok
    \uses{lem:antisymmetric-in-dlo}
\end{proof}

\begin{lemma}
    \label{lem:models-of-dlo-realize-total}
    \lean{FirstOrder.Language.Order.realize_total_of_model_dlo}
    \leanok
    Models of the theory DLO realize the totality sentence.
\end{lemma}
\begin{proof}
    \leanok
    \uses{lem:total-in-dlo}
\end{proof}

\begin{lemma}
    \label{lem:models-of-dlo-models-of-linear-order-theory}
    \lean{FirstOrder.Language.Order.model_linearOrderTheory_of_model_dlo}
    \leanok
    Models of DLO are also models of the theory of linear orders.
\end{lemma}
\begin{proof}
    \leanok
    \uses{lem:linear-order-theory-in-dlo}
\end{proof}

%%%%

%%%

\section{Instances}
%%% Simple instances and instance chains

\begin{definition}
    \label{def:le-instance-ordered-language}
    \lean{FirstOrder.Language.Order.inst_LE_of_IsOrdered}
    \leanok
    Any structure of an ordered language is an ordered set.
\end{definition}

\begin{definition}
    \label{def:preorder-instance-models-of-dlo}
    \lean{FirstOrder.Language.Order.inst_Preorder_of_dlo}
    \leanok
    \uses{def:le-instance-ordered-language, lem:models-of-dlo-realize-reflexivity, lem:models-of-dlo-realize-transitivity}
    statement of your definition
\end{definition}

\begin{definition}
    \label{def:linear-order-instance-models-of-dlo}
    \lean{FirstOrder.Language.Order.inst_LinearOrder_of_dlo}
    \leanok
    \uses{def:preorder-instance-models-of-dlo, lem:models-of-dlo-realize-antisymmetric, lem:models-of-dlo-realize-total}
    statement of your definition
\end{definition}

\begin{definition}
    \label{def:binary-relational-language}
    \lean{FirstOrder.Language.IsRelational₂}
    \leanok
    A language is binary relational if it has onyl binary relation symbols.
\end{definition}

\begin{lemma}
    \label{lem:canonical-ordered-language-binary-relational}
    \lean{FirstOrder.Language.Order.instIsRelational₂}
    \leanok
    \uses{def:binary-relational-language}
    The canonical ordered language consisting of a single binary relation $\leq$ is a binary relational language.
\end{lemma}
\begin{proof}
    \leanok
\end{proof}

%%%

\section{Structures}
%%% Equality of structures for relational languages

\begin{lemma}
    \label{lem:structure-funmap-relational-language}
    \lean{FirstOrder.Language.Structure.funMap_eq_funMap_of_relational}
    \leanok
    Structures of a relational language have equal function maps. 
\end{lemma}
\begin{proof}
    \leanok
\end{proof}

\begin{lemma}
    \label{lem:equality-of-structures-relational-language}
    \lean{FirstOrder.Language.Structure.structure_eq_structure_of_relational_iff}
    \leanok
    Two structures of a relational language are equal if and only if they agree on the 
    interpretation of all relation symbols.
\end{lemma}
\begin{proof}
    \leanok
    \uses{lem:structure-funmap-relational-language}
\end{proof}

%%%

%%% Equality of structures for binary relational languages

\begin{lemma}
    \label{lem:structure-relmap-binary-relational-language}
    \lean{FirstOrder.Language.Structure.RelMap_eq_RelMap_of_relational₂}
    \leanok
    \uses{def:binary-relational-language}
    Two structures of a binary relational language have equal relation maps if and only 
    if they agree on the interpretation of all binary relation symbols.
\end{lemma}
\begin{proof}
    \leanok
\end{proof}

\begin{lemma}
    \label{lem:equality-of-structures-binary-relational-language}
    \lean{FirstOrder.Language.Structure.structure_eq_of_structure_of_relational₂_iff}
    \leanok
    \uses{def:binary-relational-language}
    Two structures of a binary relational language are equal if and only if they agree on the 
    interpretation of all binary relation symbols.
\end{lemma}
\begin{proof}
    \leanok
    \uses{lem:structure-relmap-binary-relational-language, lem:equality-of-structures-relational-language}
\end{proof}

\begin{lemma}
    \label{lem:orders-coincide-for-ordered-structures}
    \lean{FirstOrder.Language.Order.orderStructure_of_LE_of_structure}
    \leanok
    \uses{def:le-instance-ordered-language}
    statement of your lemma
\end{lemma}
\begin{proof}
    \leanok
    \uses{def:le-instance-ordered-language, 
          lem:canonical-ordered-language-binary-relational, 
          lem:equality-of-structures-binary-relational-language,
          lem:two-vector-notation}
\end{proof}

%%%


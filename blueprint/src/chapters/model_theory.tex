\section{Instances}

Mathlib already contains a proof for the categoricity of the theory of dense linear orders without end points but it is written in terms of various instances such as `DenselyOrdered`. In this section, we produce these instances for a `FirstOrder.Language.Order` structure.

%%% Simple instances

\begin{definition}
    \label{def:le-instance-language-order}
    \lean{FirstOrder.Language.Order.instLE}
    \leanok
    Any structure of the language of orders is an ordered set.
\end{definition}

\begin{lemma}
    \label{lem:order-structure-of-le}
    \lean{FirstOrder.Language.Order.orderStructure_of_LE}
    \leanok
    \uses{def:le-instance-language-order}
    The structure induced from $\le$ which is induced from another structure is equal to the original structure.
\end{lemma}
\begin{proof}
    \leanok
    \uses{def:le-instance-language-order, lem:vector-notation}
\end{proof}

\begin{lemma}
    \label{lem:instOrderedStructureOrder_formalTextbookModelTheory}
    \lean{FirstOrder.Language.Order.instOrderedStructureOrder_formalTextbookModelTheory}
    \leanok
    \uses{def:le-instance-language-order}
    By definition, the binary relation $\le$ is equal to the interpretation of the unique binary relation symbol of the language `Language.order`.
\end{lemma}
\begin{proof}
    \leanok
    \uses{lem:order-structure-of-le}
\end{proof}

\begin{lemma}
    \label{lem:preorder-dlo}
    \lean{FirstOrder.Language.Order.DLO.instPreorder}
    \leanok
    \uses{def:le-instance-language-order}
    Models of the theory of dense linear orders without end points are preorders.
\end{lemma}
\begin{proof}
    \leanok
    \uses{lem:reflexive_mem_dlo, lem:transitive_mem_dlo}
\end{proof}

\begin{lemma}
    \label{lem:partial-order-dlo}
    \lean{FirstOrder.Language.Order.DLO.instPartialOrder}
    \leanok
    \uses{lem:preorder-dlo}
    Models of the theory of dense linear orders without end points are partial orders.
\end{lemma}
\begin{proof}
    \leanok
    \uses{lem:antisymmetric_mem_dlo}
\end{proof}

\begin{lemma}
    \label{lem:linearorder-dlo}
    \lean{FirstOrder.Language.Order.DLO.instLinearOrder}
    \leanok
    \uses{lem:partial-order-dlo}
    Models of the theory of dense linear orders without end points are linear orders.
\end{lemma}
\begin{proof}
    \leanok
    \uses{lem:total_mem_dlo}
\end{proof}

\begin{lemma}
    \label{lem:no-bot-order-dlo}
    \lean{FirstOrder.Language.Order.DLO.instNoBotOrder_formalTextbookModelTheory}
    \leanok
    \uses{lem:linearorder-dlo}
    Models of the theory of dense linear orders without end points do not have a bottom element.
\end{lemma}
\begin{proof}
    \leanok
    \uses{lem:noBotOrder_mem_dlo, lem:realize_noBot}
\end{proof}

\begin{lemma}
    \label{lem:no-min-order-dlo}
    \lean{FirstOrder.Language.Order.DLO.instNoMinOrder_formalTextbookModelTheory}
    \leanok
    \uses{lem:linearorder-dlo}
    Models of the theory of dense linear orders without end points do not have a minimum element.
\end{lemma}
\begin{proof}
    \leanok
    \uses{lem:no-bot-order-dlo}
\end{proof}

\begin{lemma}
    \label{lem:no-top-order-dlo}
    \lean{FirstOrder.Language.Order.DLO.instNoTopOrder_formalTextbookModelTheory}
    \leanok
    \uses{lem:linearorder-dlo}
    Models of the theory of dense linear orders without end points do not have a top element.
\end{lemma}
\begin{proof}
    \leanok
    \uses{lem:noTopOrder_mem_dlo, lem:realize_noTop}
\end{proof}

\begin{lemma}
    \label{lem:no-max-order-dlo}
    \lean{FirstOrder.Language.Order.DLO.instNoMaxOrder_formalTextbookModelTheory}
    \leanok
    \uses{lem:linearorder-dlo}
    Models of the theory of dense linear orders without end points do not have a maximum element.
\end{lemma}
\begin{proof}
    \leanok
    \uses{lem:no-top-order-dlo}
\end{proof}

\begin{lemma}
    \label{lem:densely-ordered-dlo}
    \lean{FirstOrder.Language.Order.DLO.instDenselyOrdered_formalTextbookModelTheory}
    \leanok
    \uses{lem:linearorder-dlo}
    Models of the theory of dense linear orders without end points are densely ordered.
\end{lemma}
\begin{proof}
    \leanok
    \uses{lem:denselyOrdered_mem_dlo, lem:realize_denselyOrdered}
\end{proof}

%%%

\section{Homomorphisms}

We also need to show that maps respecting the order structure are homomorphisms in the model theoretic sense.

\begin{lemma}
    \label{lem:to-embedding}
    \lean{FirstOrder.Language.Order.toLEmbedding}
    \leanok
    \uses{def:le-instance-language-order}    
    A function which is an order embedding is also an embedding in the model theoretic sense.
\end{lemma}
\begin{proof}
    \leanok
    \uses{def:le-instance-language-order, lem:instOrderedStructureOrder_formalTextbookModelTheory, lem:vector-notation, lem:vector-notation-under-composition}
\end{proof}

\begin{lemma}
    \label{lem:to-isomorphism}
    \lean{FirstOrder.Language.Order.toLIso}
    \leanok
    \uses{def:le-instance-language-order}
    A function which is an order isomorphism is also an isomorphism in the model theoretic sense.
\end{lemma}
\begin{proof}
    \leanok
    \uses{lem:to-embedding}
\end{proof}



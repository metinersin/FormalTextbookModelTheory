
\section{Vectors}
%%% Really trivial stuff about numbers, vectors, etc.

\begin{lemma}
    \label{lem:vector-notation}
    \lean{Matrix.Vector_eq_VecNotation₂}
    \leanok
    For any vector $v : \Fin 2 \to X$, we have $v = ![v(0), v(1)]$.
\end{lemma}
\begin{proof}
    \leanok
\end{proof}

\begin{lemma}
    \label{lem:vector-notation-under-composition}
    \lean{Matrix.comp_VecNotation₂}
    \leanok
    For any function $f : X \to Y$ and elements $x_1, x_2 \in X$, we have $f \circ ![x_1, x_2] = ![f(x_1), f(x_2)]$.
\end{lemma}
\begin{proof}
    \leanok
\end{proof}

%%%

\section{Theories}
%%% Really trivial stuff about theories

\begin{lemma}
    \label{lem:reflexive_mem_dlo}
    \lean{FirstOrder.Language.reflexive_mem_dlo}
    \leanok
    The theory of dense linear orders contains the reflexive sentence.
\end{lemma}
\begin{proof}
    \leanok
\end{proof}

\begin{lemma}
    \label{lem:transitive_mem_dlo}
    \lean{FirstOrder.Language.transitive_mem_dlo}
    \leanok
    The theory of dense linear orders without end points contains the transitive sentence.
\end{lemma}
\begin{proof}
    \leanok
\end{proof}

\begin{lemma}
    \label{lem:antisymmetric_mem_dlo}
    \lean{FirstOrder.Language.antisymmetric_mem_dlo}
    \leanok
    The theory of dense linear orders without end points contains the antisymmetric sentence.
\end{lemma}
\begin{proof}
    \leanok
\end{proof}

\begin{lemma}
    \label{lem:total_mem_dlo}
    \lean{FirstOrder.Language.total_mem_dlo}
    \leanok
    The theory of dense linear orders without end points contains the total sentence.
\end{lemma}
\begin{proof}
    \leanok
\end{proof}

\begin{lemma}
    \label{lem:noBotOrder_mem_dlo}
    \lean{FirstOrder.Language.noBotOrder_mem_dlo}
    \leanok
    The theory of dense linear orders without end points contains the no bottom element sentence.
\end{lemma}
\begin{proof}
    \leanok
\end{proof}

\begin{lemma}
    \label{lem:noTopOrder_mem_dlo}
    \lean{FirstOrder.Language.noTopOrder_mem_dlo}
    \leanok
    The theory of dense linear orders without end points contains the no top element sentence.
\end{lemma}
\begin{proof}
    \leanok
\end{proof}

\begin{lemma}
    \label{lem:denselyOrdered_mem_dlo}
    \lean{FirstOrder.Language.denselyOrdered_mem_dlo}
    \leanok
    The theory of dense linear orders without end points contains the densely ordered sentence.
\end{lemma}
\begin{proof}
    \leanok
\end{proof}

\begin{lemma}
    \label{lem:realize_noBot}
    \lean{FirstOrder.Language.Relations.realize_noBot}
    \leanok
    Models of the theory of dense linear orders without end points satisfies the no bottom element sentence.
\end{lemma}
\begin{proof}
    \leanok
\end{proof}

\begin{lemma}
    \label{lem:realize_noTop}
    \lean{FirstOrder.Language.Relations.realize_noTop}
    \leanok
    Models of the theory of dense linear orders without end points satisfies the no top element sentence.
\end{lemma}
\begin{proof}
    \leanok
\end{proof}

\begin{lemma}
    \label{lem:realize_denselyOrdered}
    \lean{FirstOrder.Language.Relations.realize_denselyOrdered}
    \leanok
    Models of the theory of dense linear orders without end points satisfies the densely ordered sentence.
\end{lemma}
\begin{proof}
    \leanok
\end{proof}

%%%


